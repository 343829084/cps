\begin{savequote}[45mm]
\ascii{I'm not a great programmer; I'm just a good programmer with great habits.}
\qauthor{\ascii{- Kent Beck}}
\end{savequote}

\chapter{类设计}
\label{ch:class-design}

%%%%%%%%%%%%%%%%%%%%%%%%%%%%%%%%%%%%%%%%%%%%%%%%%%%%%%%%%%%%%%%%%%%%%%%%%%%%%%%%
\section{构造与析构}

\begin{content}

\begin{regulation}
抵制设计和实现上帝类
\end{regulation}

上帝类是一种典型的代码坏味道,程序员几乎都曾经历过被巨类所困扰的经历。\ascii{20-30}多个成员变量在成员函数中穿梭,犹如像全局变量一样难理解和控制。此外,修改这个巨类需要高超的技能,因为你根本不确定代码的修改是否会影响其他的东西。

消除这样的巨类,关键是识别出类中的主要职责,按照\ascii{SRP}原则来分解巨类,将类拆分成为一个个细小粒度的、高度可复用的、可组合的小类\footnote{详细内容可参考\ascii{Michael C. Feathers}所著的\ascii{《Working Effectively with Legacy Code》}}。

\begin{regulation}
明确\cpp{}的初始化规则
\end{regulation}

初始化永远是程序员心中的痛,抛开一些不重要的细节,其实\cpp{}语言的初始化规则非常简单。\cpp{}使用构造函数完成类初始化,而构造函数使用初始化列表完成初始化的工作。初始化列表首先调用父类的构造函数,然后按照类中定义的的成员变量的顺序依次进行初始化。

没有显式地调用父类构造函数,编译器则自动调用父类的默认构造函数;如果对应的父类没有默认构造函数,则需显式地调用特定的带参构造函数,否则编译失败。

依次类推,如果没有显式地调用成员变量的构造函数,则编译器自动地调用成员变量的默认构造函数;如果对应的成员变量没有默认构造函数,则需显式地调用特定的带参构造函数,否则编译失败。

一般地,如果已经明确需要调用默认构造函数,为了消除冗余代码,则无需显式地进行调用。

特殊地,const数据成员,引用数据成员必须在初始化列表中进行初始化。

\begin{regulation}
在初始化列表中显式地初始化所有内置数据类型的成员变量,\cpp{}语言并没有保证其初始化
\end{regulation}

基本数据类型包括整形,浮点型,指针,引用等类型。当设计一个含有基本数据类型的成员变量时,必须定义构造函数完成所有成员变量的初始化。

\ascii{TransData}必须实现对基本数据类型的成员变量的初始化,否则行为未定义。至于泛型类型\ascii{T}则要求必须存在默认构造函数,否则编译失败。

\begin{leftbar}
\begin{c++}[caption={base/TransData.h}]
#ifndef JLGOUIW12975_VBATQ75656_YRTQKS897564_NVGSSJ
#define JLGOUIW12975_VBATQ75656_YRTQKS897564_NVGSSJ

template<typename T>
struct TransData
{
   TransData();
   ~TransData();

   void confirm()
   void revert()

private:
   T values[2];
   unsigned char valid :1;
   unsigned char memoed :1;
   unsigned char unstable :1;
   unsigned char current :1;
   unsigned char :4;
};

template<typename T>
TransData<T>::TransData()
  : valid(0), memoed(0), unstable(0), current(0)
{
}

#endif
\end{c++}
\end{leftbar}

\begin{regulation}
在初始化列表中,而非在构造函数体内进行初始化
\end{regulation}

如果类类型的成员变量没有在初始化列表中进行初始化,事实上编译器调用的是默认构造函数(如果没有默认构造函数,需要显式地调用带参构造函数;否则编译错误)。

这里需要明确地区分初始化与赋值的语义。如果将类类型的初始化放在构造函数体内,则编译器首先在初始化列表中调用默认构造函数,然后再在其构造函数体内调用\ascii{operator=}重新赋值。这样的行为往往是低效的,不是我们所期望的初始化行为。

\begin{regulation}
初始化列表顺序与成员变量定义的顺序保持一致
\end{regulation}

因为编译器的初始化是按照类中成员变量的声明顺序依次进行初始化,如果违背这个规则,可能出现初始化顺序依赖的问题。

\begin{regulation}
除非确认编译器能够正确地调用对应父类或成员变量的默认构造函数,并符合预期行为;否则需要显式地调用其带参构造函数完成初始化
\end{regulation}

相反地,如果已经确认编译器能够正确地调用对应父类或成员变量的默认构造函数,并符合预期行为,无需再显示地调用其默认构造函数,否则将产生冗余的代码。

\begin{regulation}
使用\ascii{RAII}完成资源的安全管理
\end{regulation}

\ascii{RAII(Resource acquisition is initialization)}是一种有用的技术,可实现了资源的安全管理功能。

\begin{leftbar}
\begin{c++}[caption={thread/Lock.h}]
#ifndef ITUYIYJL9_067251067_MLBNHKEIOYADMPUO0_NVAXZ
#define ITUYIYJL9_067251067_MLBNHKEIOYADMPUO0_NVAXZ

#include "thread/mutex.h"

struct Lock
{
    Lock(Mutex &mutex) : mutex(mutex)
    {
        mutex.lock();
    }

    ~Lock() 
    {
        mutex.unlock();
    }

private:
    Mutex& mutex;
};

#define SYNCHRONIZED(mutex) Lock mutex##_lock(mutex)

#endif
\end{c++}
\end{leftbar}

\ascii{RAII}利用了局部对象的自动销毁的机制,实现资源的自动释放的功能。

\begin{leftbar}
\begin{c++}[caption={thread/BlockQueue.h}]
#ifndef TYOQAMGH2975639_NGGQAOZPY96493950_GFRQA0954
#define TYOQAMGH2975639_NGGQAOZPY96493950_GFRQA0954
    
template<typename T>
void BlockQueue<T>::push(const T& item)
{
    SYNCHRONIZED(mutex);

    // others
}

#endif
\end{c++}
\end{leftbar}

\begin{advise}
在特定场景下延迟初始化对象
\end{advise}

在延迟初始化之前,必须预先占有对象所需要的原始内存空间。但被延迟的对象类型如果在如下两种场景,则无法预先占有原始内存。

\begin{enum}
  \eitem{\ascii{union}内部的放置了非\ascii{POD}类型\footnote{\cpp{}\ascii{11}已经增强了\ascii{union}的语义,\ascii{union}内可放置非\ascii{POD}类型的成员变量了。}}
  \eitem{被延迟初始化的对象无默认构造函数可调用}
\end{enum}

一般地,可以通过定义默认构造函数预先占用好对象内存的空间,当延迟初始化时再重新赋予新的值。但这完全不符合延迟初始化的概念,延迟初始化目标就是为了完全避免这次无谓的默认构造的开销。

可以引入类\ascii{Placement},完成内存的预先占用,当延迟初始化发生时,在此\ascii{Placement}的原始内存上调用\ascii{placement new}完成对象的延迟构造;并在此原始内存上提供了一层直观的、人性化的操作接口,从而可以很方便地操作被修饰的类型\ascii{T}对象。

\begin{leftbar}
\begin{c++}[caption={base/Placement.h}]
#ifndef LKSDFLKE95794_HGYETQ1240NBAZUTY75444_MGHHHH
#define LKSDFLKE95794_HGYETQ1240NBAZUTY75444_MGHHHH

#include <string.h>
#include <new>

template <typename T>
struct Placement
{
    void* alloc()
    {
        ::memset(u.buff, 0x00, sizeof(T));
        return u.buff;
    }

    void dealloc()
    {
        getObject()->~T();
    }    

    T* operator->() const
    {
        return (T*)u.buff;
    }

    T& operator*() const
    {
        return *(T*)u.buff;
    }

    T* getObject() const
    {
        return (T*)u.buff;
    }

private:
    union
    {
        char   c;
        short  s;
        int    i;
        long   l;
        float  f;
        double d;
        void*  p;

        char buff[sizeof(T)];
    }u;
};

#endif
\end{c++}
\end{leftbar}

我们分别看两个例子。第一个例子,如果存在两个非\ascii{POD}类型,它们被放在\ascii{union}中,只有在延迟初始化时才能决定其真正的类型的构造。

\begin{leftbar}
\begin{c++}[caption={money/AccountFactory.h}]
#ifndef QTA85648888_BFGSE9980666_NNNN_YERQ111_AAAA
#define QTA85648888_BFGSE9980666_NNNN_YERQ111_AAAA

#include <base/BaseTypes.h>

struct Account;

struct AccountFactory
{
    static Account* create(const U8 type);
};

#endif
\end{c++}    
\end{leftbar}

\begin{leftbar}
\begin{c++}[caption={money/AccountFactory.cpp}]
#include <money/LocalAccount.h>
#include <money/RemoteAccount.h>
#include <new>

namespace
{
    union AccountMemery
    {
        Placement<LocalAccount>  local;
        Placement<RemoteAccount> remote;
    } m;
}

Account* AccountFactory::create(const U8 type)
{
    switch(type)
    {
    case LOCAL_ACCOUNT: 
        return new (m.local.alloc()) LocalAccount;
    case REMOTE_ACCOUNT: 
        return new (m.remote.alloc()) RemoteAccount;
    default: break;
    }
    
    return 0;
}
\end{c++}    
\end{leftbar}

如果一个类包含一个类类型的数组,除非提供数组元素类型的默认构造函数,否则编译失败。使用\ascii{Placement}预先占用数组所需的内存,再在合适的时机再进行延迟初始化,则可以避免了数组元素的默认构造,提升了效率}。

\begin{leftbar}
\begin{c++}[caption={sensor/SensorRepository.h}]
#ifndef MMMM_66310_JFLA98301OQG7492_MFFAZXXX_RREE
#define MMMM_66310_JFLA98301OQG7492_MFFAZXXX_RREE
    
#include <base/Placement.h>
#include <radio/Sensor.h>
    
struct SensorRepository
{
    SensorRepository() : num(0)
    {}
    
    ~SensorRepository()
    {
        for(U16 i=0; i<num; i++)
        {
            sensors[i].~Sensor();
        }
    }

    void addSensor(const U8 id)
    {
        if(num < MAX_NUM_SENSOR)
        {
            new (sensors[num++].alloc()) Sensor(id);
        }
    }
      
private:
    enum { MAX_NUM_SENSOR = 1024 };
    
    U16 num;
    Placement<Sensor> sensors[MAX_NUM_SENSOR];
};

#endif
\end{c++}    
\end{leftbar}

有了\ascii{Placement},此时\ascii{Sensor}就没有必要定义默认构造函数,在延迟初始化的时调用真正的带参构造函数。

\begin{regulation}
绝不\ascii{public}内部的成员变量
\end{regulation}

如果公开了内部成员变量,则违背了信息隐藏机制,这是一种典型的坏味道。相反地,当遇到需要操作遗留系统中\clang{}语言的结构体,需要对其进行封装,并提供相应的成员函数,让其更加符合面向对象的思维。

\begin{leftbar}
\begin{c++}[caption={base/StructWrapper.h}]
#ifndef IIU11905749_BVFATQ1_NGFJE8563MCVYYYQQQ_NFG666
#define IIU11905749_BVFATQ1_NGFJE8563MCVYYYQQQ_NFG666
    
template <typename From, typename To>
struct StructWrapper : protected From
{
    static const To& by(const From& from)
    {
        return (const To&)from;
    }
    
    static To& by(From& from)
    {
        return (To&)from;
    }
};

#define STRUCT_WRAPPER(To, From) struct To : StructWrapper<From, To>

#endif
\end{c++}
\end{leftbar}

使用\ascii{StructWrapper}提供的静态工厂方法\ascii{by},可方便地实现\clang{}结构体转变为封装了的\cpp{}类。

%\begin{regulation}
%遇到多个构造器参数时考虑用构建器
%\end{regulation}

\begin{advise}
考虑使用静态工厂方法代替构造函数
\end{advise}

\ascii{static factory method}具有很多优势,
\begin{enum}
  \eitem{\ascii{具有比构造函数更丰富的名称}}
  \eitem{\ascii{不必每次调用时都创建一个对象,实现对象复用,提升性能}}
  \eitem{\ascii{返回类型为接口类型,遵循按接口编程的良好设计原则}}
\end{enum}

以实现对美元、美分打印的为例来讲解\ascii{static factory method}的使用。注意美元的符号在前,而美分的符号在后。
\begin{enum}
  \eitem{\ascii{532 dollars => \$532}}
  \eitem{\ascii{1030 cents => 1030¢}}
\end{enum}

在面对这个问题时,抽象出了\ascii{UsdUnit}类,并提供两个\ascii{static factory method},以实现对\ascii{UsdUnit}实例生成的控制,并通过宏定义改善了\ascii{API}的可读性和方便性。

\begin{leftbar}
\begin{c++}[caption={money/UsdUnit.h}]
#ifndef GGG612990097_BSAWQQQQ_NFGGG_VXXZZZ_76555
#define GGG612990097_BSAWQQQQ_NFGGG_VXXZZZ_76555

#include "base/Keywords.h"
#include "money/Amount.h"
#include <ostream>

DEFINE_ROLE(UsdUnit)
{
    static const UsdUnit& dollar();
    static const UsdUnit& cent();

    ABSTRACT(void format(std::ostream& oss, const Amount amount) const);
};

#define DOLLAR UsdUnit::dollar()
#define CENT   UsdUnit::cent()

#endif
\end{c++}
\end{leftbar}

\begin{leftbar}
\begin{c++}[caption={money/UsdUnit.cpp}]
#include "money/UsdUnit.h"

namespace
{
    struct dollar_class : UsdUnit
    {
        OVERRIDE(void format(std::ostream& oss, const Amount amount) const)
        {
            oss << "$" << amount;
        }
    };

    struct cent_class : UsdUnit
    {
        OVERRIDE(void format(std::ostream& oss, const Amount amount) const)
        {
            oss << amount << "¢";
        }
    };
}

#define DEF_USD_UNIT(unit)             \
const UsdUnit& UsdUnit::unit()         \
{                                      \
    static unit##_class inst;          \
    return inst;                       \
}

DEF_USD_UNIT(dollar)
DEF_USD_UNIT(cent)
\end{c++}
\end{leftbar}

\begin{advise}
通过私有构造函数强化不可实例化能力
\end{advise}

此处提供了一个\ascii{String}的工具类,为了保证禁止生成实例的约束,提供了一个故意没有实现的默认构造函数,从而保证了编译时的安全性。

\begin{leftbar}
\begin{c++}[caption={util/StringUtil.h}]
#ifndef UURY08453HGYT6574_NBBVGA_EYQIGYFH_GWQAMGY
#define UURY08453HGYT6574_NBBVGA_EYQIGYFH_GWQAMGY

#include <vector>
#include <string>

struct StringUtil
{
    static std::vector<std::string> split
        (const std::string &text, char separator);

    static std::string wrap
        (const std::string &text, int wrapColumn = 80);

private:
    StringUtil();
};

#endif
\end{c++}
\end{leftbar}

但在实际操作中,常常不这么啰嗦。因为诸如类似的工具类,没有人会通过实例调用静态方法:

反例:
\begin{leftbar}
\begin{c++}
StringUtil().split("1,2,3", ',');
\end{c++}
\end{leftbar}

此时需要在编译安全性及其容忍冗余代码之间寻找平衡点。如果用户存在大概率地犯错的可能性,或做最保守的防御式编程,则应明确地拒绝;相反如果你的团队成员平均能力已经达到很高层次,日常\ascii{Code Review}的频率也比较高,则冗余的代码反而成为眼中刺。

是否需要将复制构造函数或拷贝赋值函数显式地声明为\ascii{private},道理是一样的。

\begin{advise}
当一个类确定需要禁止被复制和赋值时,应明确地拒绝
\end{advise}

为了防止其他人错误地进行值拷贝,常常需要禁止其复制和赋值行为,存在三种典型的技术:

\begin{leftbar}
\begin{c++}[caption={base/Uncloneable.h}]
#ifndef CCCPLSUE746HNV_NGGWTEEEYYY11857664_NFFFGGG
#define CCCPLSUE746HNV_NGGWTEEEYYY11857664_NFFFGGG

#define DISALLOW_COPY(classname) \
private:                         \
    className(const className&)
    
#define DISALLOW_ASSIGN(classname) \
private:                           \
    className& operator=(const className&);
    
#define DISALLOW_COPY_AND_ASSIGN(className) \
    DISALLOW_COPY(classname)                \
    DISALLOW_ASSIGN(classname)
    
#endif
\end{c++}
\end{leftbar}

或\ascii{private}继承\ascii{Uncloneable}类:

\begin{leftbar}
\begin{c++}[caption={base/Uncloneable.h}]
#ifndef CCCPLSUE746HNV_NGGWTEEEYYY11857664_NFFFGGG
#define CCCPLSUE746HNV_NGGWTEEEYYY11857664_NFFFGGG

class Uncloneable
{
protected:
    Uncloneable() {}
    ~Uncloneable(){}

private:
    Uncloneable(const Uncloneable&);
    Uncloneable& operator=(const Uncloneable&);
};

#endif
\end{c++}
\end{leftbar}

在\cpp{}\ascii{11}中,可以显式地\ascii{delete}掉相应的函数:
\begin{leftbar}
\begin{c++}[caption={cppunit/Test.h}]
#ifndef UTYWO_TETQHBF774010_NVBDAUTHHF_HH1HHJFTUT_144
#define UTYWO_TETQHBF774010_NVBDAUTHHF_HH1HHJFTUT_144
    
struct Test
{
    Test(const Test&) = delete;
    Test& operator=(const Test&) = delete;
};

#endif
\end{c++}
\end{leftbar}

但此规则只会在很少的情况下才被使用。因为在实际代码中,我们几乎都不应该进行值拷贝。除非该类的对象很容易被人错误地进行值拷贝,才需要显示地禁止其拷贝复制的行为。

\begin{regulation}
避免创建不必要的对象
\end{regulation}

对象创建和销毁是有代价的,尤其在关乎性能的对象创建和销毁时,需要特别地关注。

这时出现了很多技术来解决这方面的问题,对象池技术和\ascii{Flyweight}模式是最常见的技术,例如线程池,数据库连接池,网络连接池等等。

参考实例可参见\ascii{Joshua Bloch}所著的\ascii{《Effective Java》},规则同样适用于\cpp{}。

\end{content}

\section{继承与多态}

\begin{content}

\begin{regulation}
按接口编程
\end{regulation}

按接口编程是面向对象中最重要的原则之一,因为对业务抽象的接口往往体现了概念间的契约关系。但在\cpp{}语言中,提供接口最让人繁琐的一件事情是:需要显式地提供一个\ascii{virtual}析构函数。

可以通过\ascii{DEFINE\_ROLE}的宏定义来实现对接口的定义,从而可以消除子类对虚拟析构函数的重复实现。

\begin{leftbar}
\begin{c++}[caption={base/Role.h}]
#ifndef HF95EF112_D6C6_4DB0_8C1A_BE5A6CF8E3F1
#define HF95EF112_D6C6_4DB0_8C1A_BE5A6CF8E3F1
#include <base/Keywords.h>

namespace details
{
   template <typename T>
   struct Role
   {
      virtual ~Role() {}
   };
}

#define DEFINE_ROLE(type) struct type : ::details::Role<type>

#endif
\end{c++}
\end{leftbar}

\begin{leftbar}
\begin{c++}[caption={thread/Callable.h}]
#ifndef NVBB1QQAT1838_BZZZAAAIIRUUYWWW887440_HHFF_123
#define NVBB1QQAT1838_BZZZAAAIIRUUYWWW887440_HHFF_123

#include "base/Role.h"

DEFINE_ROLE(Callable)
{
    ABSTRACT(void call());
};

#endif
\end{c++}
\end{leftbar}

\begin{regulation}
避免从并非设计为基类的类中继承
\end{regulation}

直到\cpp{}\ascii{11}标准才引入了类似于\ascii{Java}语言的\ascii{final}关键字,以便显式地阻止子类化。通常情况下,所有的\cpp{}类都可以被继承。但存在如下几个场景,类应该设计为\ascii{final}类:

\begin{enum}
  \eitem{并非为了安全地进行子类化而设计的类}
  \eitem{设计不可变类,保证多线程的安全性}
  \eitem{\ascii{final}类为编译器的优化提供更多的线索}
\end{enum}

如果一个类在设计之初,已经决定是\ascii{final}类,可以明确地告诉编译器。这里可以使用宏定义,即使你的编译器还不支持\cpp{}\ascii{11},通过\ascii{FINAL},表达对应的\ascii{virtual}函数不能再被覆写(\ascii{override})或类不能被子类化,以便增强表现力。

\begin{leftbar}
\begin{c++}
#include <base/Config.h>

#if __SUPPORT_FINAL
# define FINAL final
#else
# define FINAL
#endif
\end{c++}
\end{leftbar}

因为历史原因,存在有些类并不是以子类化为目的而设计的,但是这些类往往拥有\ascii{public non-virtual}析构函数,当子类的析构函数需要释放特定资源时,将发生未定义的行为。

如果你试图继承\ascii{string}或\ascii{STL}的其他容器,将走上这条不归路。

\begin{regulation}
为不具有多态性的基类提供\ascii{protected non-virtual}析构函数
\end{regulation}

一个类可被子类化,但不具有多态性,明确地加上\ascii{protected non-virtual destructor}可改善设计的意图,提高表现力。

可惜的是,这条规则往往没有得到社区的普遍认识。当出现这样的场景,大家更习惯了让编译器自动生成一个默认的\ascii{public non-virtual destructor}。所以,同样需要你在提高编译安全性与容忍冗余代码之间进行权衡和考虑。

标准库中的\ascii{std::input\_iterator\_tag, std::unary\_function}等类就是不具有多态性质的基类。

\begin{regulation}
考虑使用\ascii{virtual}函数声明\ascii{private},而将\ascii{public}函数声明\ascii{non-virtual}
\end{regulation}

此方法常被称为\ascii{NVI(non-virtual interface)},它是在\cpp{}语言中实现\ascii{Template Method}模式的独特表现形式。\ascii{Template Method}在父类中以一个\ascii{public}接口实现算法的骨架,并以\ascii{private virtual}挂接一个或多个回调钩子的成员函数供子类覆写。

\begin{regulation}
区分\ascii{pure virtual, virtual, non-virtual}函数的语义
\end{regulation}

当定义一个类时,清晰、准确的使用这三个特性,有助于其他人理解类设计的意图。

\begin{enum}
  \eitem{\ascii{pure virtual:}为了让子类只继承其接口}
  \eitem{\ascii{virtual:}让子类继承接口和一份默认的实现}
  \eitem{\ascii{non-virtual:}让子类继承其接口和一份强制性的实现}
\end{enum}

\begin{regulation}
改写\ascii{virtual}函数时,必须显式地标示\ascii{override}
\end{regulation}

在子类中改写虚函数时,提供显式的\ascii{override}将改善代码的可读性,更重要的是增强了编译时的安全性检查。

\begin{leftbar}
\begin{c++}[caption={cppunit/TestDecorator.h}]
#ifndef NCQEAY2220098666TTRRWWO_WWT2264449880MM_SDQAA
#define NCQEAY2220098666TTRRWWO_WWT2264449880MM_SDQAA

#include <cppunit/core/Test.h>

struct TestDecorator : Test
{
    explicit TestDecorator(Test& test);

private:
    OVERRIDE(int countTestCases() const);
    OVERRIDE(std::string getName() const);
    OVERRIDE(void run(TestResult& result);
    OVERRIDE(int getChildTestCount() const);

private:
    Test& test;
};

#endif
\end{c++}
\end{leftbar}

但可惜\ascii{override}关键字直到\cpp{}\ascii{11}才被列入标准,但可以提供\ascii{OVERRIDE}宏定义实现代码的可移植性,即使你的编译器不支持\cpp{}\ascii{11},也可以改善代码的表现力。

\begin{leftbar}
\begin{c++}
#include <infra/base/Config.h>

#if __SUPPORT_VIRTUAL_OVERRIDE
# define OVERRIDE(...) virtual __VA_ARGS__ override
#else
# define OVERRIDE(...) virtual __VA_ARGS__
#endif
\end{c++}
\end{leftbar}

\begin{regulation}
为多态基类提供\ascii{virtual destructor}
\end{regulation}

何时需要给类增加\ascii{virtual destructor}常常让人抓狂,请记住两条规则:

\begin{enum}
  \eitem{\ascii{As base class}}
  \eitem{\ascii{Polymorphic}}
\end{enum}

这两个条件必须同时满足,才可以为类增加\ascii{virtual destructor}。一般地,如果一个类包含虚函数时,就为它增加\ascii{virtual destructor}。

但需要注意的是,为子类化为目的,但没有运行时多态的语义时,\ascii{Herb Sutter, Andrei Alexandrescu}建议显式地定义\ascii{protected non-vritual destructor}。

\begin{regulation}
禁止在\ascii{constructor, destructor}中调用虚函数
\end{regulation}

在\ascii{constructor, destructor}调用期间,子类对象尚未构造,或已经销毁,所以在其父类的\ascii{constructor, destructor}中调用虚函数发生多态行为不是我们所期望的,所以禁止在\ascii{constructor, destructor}中调用虚函数。

\begin{regulation}
绝不重定义继承而来的\ascii{non-virtual}函数;也绝不重定义继承而来的缺省参数值
\end{regulation}

这是因为其重定义没有发生期望的多态行为,所以被禁止使用,以免产生反直觉的行为。

\begin{regulation}
避免遮掩继承而来的名称
\end{regulation}

这是\ascii{C++}在嵌套作用域的名字隐藏机制在类作用域中的一个具体表现形式。当出现这些特殊的异常的情况时,往往能嗅探出不良命名和设计的坏味道。

\begin{regulation}
使用函数对象表示策略
\end{regulation}

策略模式是一种最常见的设计模式之一,\cpp{}语言中,常常使用函数对象、或定义多态策略的接口来实现策略的多态变化。

例如\ascii{std::sort}中实现的排序,使用\ascii{Comparator}方便用户定制比较规则。

\begin{leftbar}
\begin{c++}[caption={标准库std::sort}]
template<typename Iterator, typename Comparator>
void sort(Iterator first, Iterator last, Comparator comp);
\end{c++}
\end{leftbar}

\end{content}
