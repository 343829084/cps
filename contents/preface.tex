\chapter{前言} 
\label{ch:preface}

\begin{content}

\cpp{} \ascii{Programming Style}是一份关于\cpp{}的代码规范,为\cpp{}程序员提供编码的参考和建议。

规范总结了之前实践中遇到的一些宝贵的经验教训,并且很多规则、原则和建议都摘自社区软件设计大师的论文、报告、书籍、博客,并引用了一些\ascii{open source}典型源代码。

由于时间的仓促,本规范只归纳了部分的典型\cpp{}原则、规则和意见。如果本规范有您不清楚的地方,请查阅附录中的参考文献。

\end{content}

\section*{定义}

\begin{content}

\begin{table}[!htb]
\resizebox{0.95\textwidth}{!} {
\begin{tabular*}{1.2\textwidth}{@{}ll@{}}
\toprule
\ascii{约定} & \ascii{说明} \\
\midrule
\ascii{原则}  & 坚持遵守的指导思想 \\
\ascii{规则} & 必须遵守的约定 \\ 
\ascii{建议} & 可以考虑的约定 \\ 
\ascii{正例} & 原则、规则、建议所给出的正确例子 \\ 
\ascii{反例} & 原则、规则、建议所给出的错误例子 \\ 
\bottomrule
\end{tabular*}
}
\caption{规范约定}
\label{tbl:regulation-tbl}
\end{table}

\end{content}

\section*{宏}

\begin{content}

为了提高代码的表现力,规范中使用了一部分实用的宏定义。

\begin{leftbar}
\begin{c++}[caption={base/Default.h}]
#ifndef GKOQWPRT_1038935_NCVBNMZHJS_8909603
#define GKOQWPRT_1038935_NCVBNMZHJS_8909603

namespace details
{
   template <typename T>
   struct DefaultValue
   {
      static T value()
      {
         return T();
      }
   };

   template <typename T>
   struct DefaultValue<T*>
   {
       static T* value()
       {
           return 0;
       }
   };

   template <typename T>
   struct DefaultValue<const T*>
   {
       static T* value()
       {
           return 0;
       }
   };

   template <>
   struct DefaultValue<void>
   {
      static void value()
      {
      }
   };
}

#define DEFAULT(type, method)  \
    virtual type method { return ::details::DefaultValue<type>::value(); }

#endif
\end{c++}
\end{leftbar}

\ascii{DEFAULT}对于定义空实现的\ascii{virtual}函数非常方便。需要注意的是,所有计算都是发生在编译时的。

\begin{leftbar}
\begin{c++}[caption={base/Keywords.h}]
#ifndef H16274882_9153_4DB2_A2E2_F23D4CCB9381
#define H16274882_9153_4DB2_A2E2_F23D4CCB9381

#include "base/Config.h"
#include "base/Default.h"

#define ABSTRACT(...) virtual __VA_ARGS__ = 0

#if __SUPPORT_VIRTUAL_OVERRIDE
#   define OVERRIDE(...) virtual __VA_ARGS__ override
#else
#   define OVERRIDE(...) virtual __VA_ARGS__
#endif

#define EXTENDS(...) , ##__VA_ARGS__
#define IMPLEMENTS(...) EXTENDS(__VA_ARGS__)

#endif
\end{c++}
\end{leftbar}

\ascii{Config.h}提供了编译器支持\cpp{}\ascii{11}特性的配置信息。\ascii{ABSTRACT, OVERRIDE, EXTENDS, IMPLEMENTS}等关键字,使得\ascii{Java}程序员也能看懂规范中\cpp{}的代码,也极大地改善了\cpp{}的表现力。

\begin{leftbar}
\begin{c++}[caption={base/Role.h}]
#ifndef HF95EF112_D6C6_4DB0_8C1A_BE5A6CF8E3F1
#define HF95EF112_D6C6_4DB0_8C1A_BE5A6CF8E3F1
#include <base/Keywords.h>

namespace details
{
   template <typename T>
   struct Role
   {
      virtual ~Role() {}
   };
}

#define DEFINE_ROLE(type) struct type : ::details::Role<type>

#endif
\end{c++}
\end{leftbar}

通过\ascii{DEFINE\_ROLE}的宏定义来实现对接口的定义,从而可以消除子类对虚拟析构函数的重复实现。

\end{content}

\section*{致谢}

\begin{content}

本规范主要由刘光聪编写,并由如下同事:黄新文,刘苏建,王妮,曾亮亮,李亚军,李永顺,及其\ascii{LTE}开发部的部分同事共同参与审阅、评审、修改和代码走查,并提出了非常好的修改意见。

另外深圳\ascii{LTE}开发部的部分同事,西安基带部的王立平、李小海也参与了规范的评审,并提出了宝贵的意见。

\end{content}