\begin{savequote}[45mm]
\ascii{Premature optimization is the root of all evil.}
\qauthor{\ascii{– Donald Knuth}}

\ascii{On the other hand, we cannot ignore efficiency.}
\qauthor{\ascii{– Jon Bentley}}
\end{savequote}

\chapter{函数设计}
\label{ch:functions-operators}

%%%%%%%%%%%%%%%%%%%%%%%%%%%%%%%%%%%%%%%%%%%%%%%%%%%%%%%%%%%%%%%%%%%%%%%%%%%%%%%%
\section{函数}

\begin{content}

\begin{regulation}
避免过长,嵌套过深的函数实现
\end{regulation}

我讨厌长函数,犹如讨厌重复一样。长函数往往伴随着复杂的逻辑判断,过深嵌套逻辑。

但也必要硬性地规定团队函数不能超过固定的行,因为这个规则很容易被打破。如果大家都遵循良好的设计原则,养成良好的职责分离的设计的基本素养,实现出来的函数大多平均都在\ascii{5}行以内,或者更少。这绝对不是理想,已经存在众多的、典型的成功案例。

\begin{regulation}
只做一件事,并做好这件事
\end{regulation}

这是\ascii{SRP}在函数实现中的一个具体体现。一个函数只应该承担一个唯一的职责,如果函数名伴随\ascii{and},或将查询和命令混合往往是违背此原则的信号。

一般地,如果函数只是做了该函数名称下统一抽象层次上的几个小步骤,则函数还是只做了一件事情。函数就是将一个大一点的概念拆分成级别稍微低一点、并在同一个抽象层次的一系列步骤的过程。

\begin{regulation}
函数中的每一个语句都在一个相同的抽象层次上
\end{regulation}

如果函数中混杂不同抽象层次的代码,往往让人感到迷惑,理解代码的逻辑,需要读者陷入到细节之中而不能自拨。

\ascii{Kent Beck}建议使用\ascii{Compose Method}分解长函数;\ascii{Martin Flower}也建议使用\ascii{Extract Method}进行函数提取。\ascii{Extract Method}最重要的就是识别出代码中的抽象层次,并梳理出主干,按照自顶向下的规则实现函数的。

函数提取常常要遵守如下3个基本原则:
\begin{enum}
  \eitem{\ascii{在同一个抽象层次}}
  \eitem{\ascii{给一个直观的,意图明确的好名字}}
  \eitem{\ascii{实现对称性}}
\end{enum}

正如\ascii{Bob}大叔提到的那样: 程序就像是一系列的以\ascii{TO}起头的段落,每一段落都描述了当前抽象层次的逻辑,并引用下一抽象层次的,以\ascii{TO}开头的段落。

下例虽然以\ascii{Java}为例,但重点没有偏离,规则同样适用于\clang{}/cpp{}。

\begin{leftbar}
\begin{java}[caption={java.util.List}]
public class List<E> {
   public void add(E element) {
      if (!readOnly) {
         int newSize = size + 1;
         if (newSize > elements.length) {
            Object[] newElements =
               new Object[elements.length + 10];
            for (int i = 0; i < size; i++)
               newElements[i] = elements[i];
            elements = newElements;
         }
         elements[size++] = element;
      }
   }
}
\end{java}
\end{leftbar}

提取函数之后,使算法的骨骼显而易见。

\begin{leftbar}
\begin{java}[caption={java.util.List}]
public class List<E> {
   public void add(E element) {
      if (readOnly)
         return;
      
      if (atCapacity())
         grow();
      
      addElement(element);
   }
}
\end{java}
\end{leftbar}

提取函数往往受到性能偏执狂的抨击,他们认为过多的函数提取将成为性能的主要瓶颈。\ascii{2-8}原则将彻底击垮所有不靠谱的言论,与其纠结于函数调用的开销,不如花费更多的时间去优化和改善那些至关重要的算法和关键代码。

\begin{regulation}
检查参数的有效性
\end{regulation}

绝大部分的函数对于传递的参数都有限制,这时需要在函数执行之前完成参数的合法性校验。\clang{}/cpp{}虽不能天然地支持“按契约设计”,但前置的参数检查将大大改善程序的健壮性。

\begin{regulation}
分离查询与指令
\end{regulation}

函数应该只拥有清晰明确的、唯一的职责。如果函数既修改对象状态,又返回对象的状态信息,则常常会导致混乱。产生副作用的指令操作,都是系统状态的一种变更,将产生时序上的耦合和顺序的依赖。尤其当查询和指令混合在一起的时候,查询函数的无副作用的优点将不复存在;如果一个函数名本身具有查询语义,但混合了指令操作,将产生更大的反直觉。

\begin{regulation}
使用\ascii{Null Object}替代空指针
\end{regulation}

校验空指针是一件及其繁琐的事情,优秀的程序员通过各种技巧绕开这些冗余的操作。例如参数通过引用传递,另外一种常见的手段就是\ascii{Null Object}\footnote{参考\ascii{Martin Flower}的著作《重构,改善既有代码的设计》}。

\begin{regulation}
对于参数类型,返回值类型,优先使用接口类型
\end{regulation}

遵循按接口编程的良好习惯,是优秀设计师的基本素养。

\begin{regulation}
对于\ascii{bool}参数,优先使用两个元素的枚举类型
\end{regulation}

从直观上,

\begin{leftbar}
\begin{c++}
Thermometer::newInstance(CELSIUS);
\end{c++}
\end{leftbar}

要比

\begin{leftbar}
\begin{c++}
Thermometer::newInstance(true);
\end{c++}
\end{leftbar}

容易理解得多,但牺牲了一点点代码复用性。对于这个问题,可以通过内部的\ascii{private}函数来解决这个问题。也就是说,带\ascii{bool}参数的函数依然存在,只不过被\ascii{private},以便实现对枚举参数的函数的代码复用。

\begin{regulation}
永远不要导出具有相同参数数目的的重载方法
\end{regulation}

重载是一种编译时的多态。只要函数具有相同名字,但参数数目,类型不同,即为重载。但滥用往往会误导使用\ascii{API}的程序员,尤其在具有相同参数数目的重载方法时,程序员需要清晰地知道所有类型的隐式转换规则,即其重载匹配规则,这无疑是一种没必要的负担。

\begin{leftbar}
\begin{c++}[caption={io/ObjectOutputStream.h}]
#ifndef JAAA_1295_BBVBAA_GFYQQPPAAMZZ0092444_NDFQQAA
#define JAAA_1295_BBVBAA_GFYQQPPAAMZZ0092444_NDFQQAA

struct ObjectOutputStream
{
    void write(bool);
    void write(char);
    void write(short);
    void write(int);
    void write(long);
    void write(float);
    void write(double);
};

#endif
\end{c++}
\end{leftbar}

面对疑难的时候,抛弃重载往往能得到更不容易犯错的解决方案。

\begin{leftbar}
\begin{c++}[caption={io/ObjectOutputStream.h}]
#ifndef JAAA_1295_BBVBAA_GFYQQPPAAMZZ0092444_NDFQQAA
#define JAAA_1295_BBVBAA_GFYQQPPAAMZZ0092444_NDFQQAA

struct ObjectOutputStream
{
    void writeBool(bool);
    void writeChar(char);
    void writeShort(short);
    void writeInt(int);
    void writeLong(long);
    void writeFloat(float);
    void writeDouble(double);
};

#endif
\end{c++}
\end{leftbar}

\begin{regulation}
使用\ascii{explicit}显式地禁止类型的隐式转换
\end{regulation}

隐式类型转换往往无声无息地被执行,通过\ascii{explicit}便能通过编译器清晰地捕获的所有隐式转换的事件,以便供程序员进一步决策。

\begin{regulation}
避免实现火车失事的代码
\end{regulation}

反例:
\begin{leftbar}
\begin{java}
String outputDir = ctxt.getOptions().getScratchDir().getAbsolutePath();
\end{java}
\end{leftbar}

本例虽然使用\ascii{Java}语言编写,但这不影响对火车失事问题的理解。如果其调用链如果某一环节出现了问题,则整个调用将出现失败。这类串联的调用违反了\ascii{Demeter}法则,\ascii{ctxt}对象包含了多个选项,每个选项中存在一个临时目录,每个目录都有一个绝对路径,所有的知识都毫无保留地暴露给了用户。

仔细分析一下代码,再看看其中一个模块是如何使用这个\ascii{outputDir}的。

\begin{leftbar}
\begin{java}
String outFile = outputDir + File.SEPERATOR + className + ".class";
BufferedOutputStream bos = new BufferedOutputStream(new FileOutputStream(outFile));
\end{java}
\end{leftbar}

所有的一切都是为了创建指定名称的临时文件,理想的实现应该将所有本应该隐藏的知识归并到\ascii{ctxt}中实现。

正例:
\begin{leftbar}
\begin{c++}
BufferedOutputStream bos = ctxt.createScratchFileStream(className);
\end{c++}
\end{leftbar}

\end{content}

\section{重载操作符}

\begin{content}

\begin{regulation}
重载运算符必须保持原有操作符的语义
\end{regulation}

如果重载运算符改变了程序员对固有知识的理解,将加大放错的几率。

\begin{regulation}
谨慎地使用转换运算符
\end{regulation}

转换运算符因为其隐式转换而变得非常神秘,需谨慎使用。并并非在任何场景都不使用转换操作符,在合适的情况下使用装换运算符,能够得到更为简洁的代码。

\begin{regulation}
优先使用前置\ascii{++operator}
\end{regulation}

当重载了\ascii{++operator}的类对象,例如迭代器,更提倡使用\ascii{++operator},这是提升效率的一种举措。

\begin{regulation}
使用\ascii{operator*, operator->}实现类的包装或修饰
\end{regulation}

\ascii{operator*, operator->}操作符是提供包装和修饰功能的特殊工具,是一种典型的间接技术\footnote{软件工程有一条黄金定律:任何问题都可以通过增加一个间接层来解决。}。它的最大优势是为用户提供良好的、人性化间接操作接口。

\ascii{std::unique\_ptr, std::shared\_ptr}等智能指针,最关键的也是重载了这两个操作符,从而实现了操作智能指针犹如操作原始指针一般。

\ascii{Placment}本身也是一个包装器,只不过它修饰的是一块原始的内存。通过\ascii{operator*, operator->}可方便地操作寄存在原始内存上的\ascii{T}对象。

\ascii{AutoMsg}也是一个较为经典的例子,\ascii{AutoMsg}将消息对象的构造搬迁到了堆上进行初始化,从而避免了由于消息过大而导致的栈溢出的风险。当函数调用出栈时,内存自动释放。

\begin{leftbar}
\begin{c++}[caption={base/AutoMsg.h}]
#ifndef AA_TTYSFLNVHSL_375BBGGH7754420010MHHSLJEROLJKD
#define AA_TTYSFLNVHSL_375BBGGH7754420010MHHSLJEROLJKD

template <typename Msg>
struct AutoMsg
{
    AutoMsg() : msg(new Msg)
    {}
    
    ~AutoMsg()
    {
        delete msg;
    }
    
    Msg& operator*()
    {
        return *msg;
    }
    
    Msg* operator->()
    {
        return msg;
    }
    
private:
    Msg* msg;
};

#endif
\end{c++}
\end{leftbar}

\begin{leftbar}
\begin{c++}[caption={money/RemoteAccount.h}]
#ifndef QLALK_YRUUUTIIT00475_NVBBAKKF_6455BCHKAK_YQQAMGH
#define QLALK_YRUUUTIIT00475_NVBBAKKF_6455BCHKAK_YQQAMGH

#include <money/domain/Account.h>
#include <money/msg/TransMoneyMsg.h>
#include <base/Asserter.h>
    
struct RemoteAccount
{
    Status transMoneyTo(const Account& to)
    {
        AutoMsg<TransMoneyMsg> msg;
        
        ASSERT_SUCC_CALL(buildMsg(*msg));
        ASSERT_SUCC_CALL(sendMsg(to, *msg));
        
        return SUCCESS;
    }
    
private:
    Status buildMsg(TransMoneyMsg&)
    {
        // no implement
    }
    
    Status sendMsg(const Account& to, const TransMoneyMsg& msg)
    {
        // no implement
    }    
};

#endif
\end{c++}
\end{leftbar}

\end{content}
