\begin{savequote}[45mm]
\ascii{Write programs for people first, computers second.}
\qauthor{\ascii{- Steve McConnell}}
\end{savequote}

\chapter{命名} 
\label{ch:naming}

\begin{content}

良好的命名将改善代码的表现力,与其说命名是一门技术,不如说它是一门艺术。必须坚持、并慎重地给程序中的每一个实体取好名字,让其名副其实,代码的可读性将得到极大的改善。

\end{content}

\section{Baby Names}
\begin{content}

\begin{regulation}
遵守统一的命名规范
\end{regulation}

一些程序设计语言,从诞生的时刻就有着自己的文化背景,整个社区有着统一的命名规则。但\cpp{}在社区中存在众多的命名风格,有的与时俱进,也非常人性化;有的则已经与现代软件工程格格不入\footnote{例如社区中依然存在一批忠实的匈牙利的守护者,他们认为一个名称没有前缀和后缀标识它们的类型,他们几乎都看不懂代码了。}。

下文罗列了一些\cpp{}典型的命名规范供参考。每一种命名规范都有自身自己的优缺点,团队应该根据自身的实际情况选择适合自己的命名规范。最重要的是,团队应遵循统一的命名规范,不应该出现两种或两种以上的命名风格。

我们推荐团队使用下文中任意一种命名风格,但如果由于历史遗留原因导致革新非常困难,而无法采用新的命名规范,团队依然要保持统一的命名规范,避免出现多种命名风格。

\begin{table}[!htb]
\resizebox{0.95\textwidth}{!} {
\begin{tabular*}{1.2\textwidth}{@{}ll@{}}
\toprule
\ascii{Identifier} & \ascii{Examples} \\
\midrule
\ascii{Namespace}  & \ascii{std, dcm, mockcpp, testing} \\
\ascii{Class/Struct/Union} & \ascii{Timer, FutureTask, LinkedHashMap, HttpServlet} \\ 
\ascii{Method} & \ascii{remove, ensureCapacity, getCrc} \\
\ascii{Constant/Macro/Enum} & \ascii{IDLE, ACTIVE, MAX\_LINK\_NUM} \\
\ascii{Local Variable} & \ascii{i, xref, houseNumber} \\
\ascii{Type Parameter} & \ascii{T, E, K, V, X, T1, T2} \\
\bottomrule
\end{tabular*}
}
\caption{驼峰命名规范1}
\label{tbl:naming-1}
\end{table}

还有一种命名风格与上一种风格类同,只是成员函数/函数都以大写开头的驼峰命名。

\begin{table}[!htb]
\resizebox{0.95\textwidth}{!} {
\begin{tabular*}{1.2\textwidth}{@{}ll@{}}
\toprule
\ascii{Identifier} & \ascii{Examples} \\
\midrule
\ascii{Namespace}  & \ascii{std, dcm, mockcpp, testing} \\
\ascii{Class/Struct/Union} & \ascii{Timer, FutureTask, LinkedHashMap, HttpServlet} \\ 
\ascii{Method} & \ascii{Remove, EnsureCapacity, GetCrc} \\
\ascii{Constant/Macro/Enum} & \ascii{IDLE, ACTIVE, MAX\_LINK\_NUM} \\
\ascii{Local Variable} & \ascii{i, xref, houseNumber} \\
\ascii{Type Parameter} & \ascii{T, E, K, V, X, T1, T2} \\
\bottomrule
\end{tabular*}
}
\caption{驼峰命名规范2}
\label{tbl:naming-2}
\end{table}


第三种命名风格,主要体现在标准库或\ascii{Boost}准标准库社区,其规则非常简单,下划线分割的全小写,或下划线分割的全大写。

\begin{table}[!htb]
\resizebox{0.95\textwidth}{!} {
\begin{tabular*}{1.2\textwidth}{@{}ll@{}}
\toprule
\ascii{Identifier} & \ascii{Examples} \\
\midrule
\ascii{Namespace}  & \ascii{boost, details, mpl} \\
\ascii{Class/Struct/Union} & \ascii{any, is\_enum, shared\_ptr} \\ 
\ascii{Method} & \ascii{any\_cast, type\_of} \\
\ascii{Constant/Macro/Enum} & \ascii{IDLE, ACTIVE, MAX\_LINK\_NUM} \\
\ascii{Local Variable} & \ascii{i, xref, house\_number} \\
\ascii{Type Parameter} & \ascii{T, E, K, V, X, T1, T2} \\
\bottomrule
\end{tabular*}
}
\caption{标准库或\ascii{boost}命名规范}
\label{tbl:naming-3}
\end{table}

\begin{regulation}
绝不使用汉语拼音命名
\end{regulation}

必须实用标准的英语命名程序实体,不允许实用汉语拼音。

\begin{regulation}
类名应该是名词或名词短语;接口可以是形容词; 方法名应该是动词或动词短语
\end{regulation}

\begin{table}[!htb]
\resizebox{0.95\textwidth}{!} {
\begin{tabular*}{1.2\textwidth}{@{}ll@{}}
\toprule
\ascii{类别} & \ascii{举例} \\
\midrule
\ascii{正确的类名}  & \ascii{AddressParser, EventRegistry} \\
\ascii{错误的类名} & \ascii{ParseAddress, RegisterEvent} \\
\bottomrule
\end{tabular*}
}
\caption{类名}
\label{tbl:naming-4}
\end{table}

接口可以是形容词,最为常见的就是定义能力接口,即以\ascii{-able}结尾。

反例:
\begin{leftbar}
\begin{c++}[caption={\ttfamily{cub/thread/Runnable.h}}]
#ifndef LJKYIUERUIQ09857983_NVHKJKHA86100494_JGYQPIPWNC
#define LJKYIUERUIQ09857983_NVHKJKHA86100494_JGYQPIPWNC

#include <cub/base/Keywords.h>

// 接口必须是形容词或名词,不允许是动词
DEFINE_ROLE(Run)
{
    ABSTRACT(void run());
};

#endif
\end{c++}
\end{leftbar}

正例:
\begin{leftbar}
\begin{c++}[caption={\ttfamily{cub/thread/Runnable.h}}]
#ifndef LJKYIUERUIQ09857983_NVHKJKHA86100494_JGYQPIPWNC
#define LJKYIUERUIQ09857983_NVHKJKHA86100494_JGYQPIPWNC

#include <cub/base/Keywords.h>

DEFINE_ROLE(Runnable)
{
    ABSTRACT(void run());
};

#endif
\end{c++}
\end{leftbar}

\begin{regulation}
如果函数返回值为\ascii{bool},加上\ascii{is, has, can, should, need}将会增强语意
\end{regulation}

反例:
\begin{leftbar}
\begin{c++}
bool readPassword = true;
\end{c++}
\end{leftbar}

至少存在两种解释,

\begin{enum}
  \eitem{\ascii{We need to read the password}}
  \eitem{\ascii{The password has already been read}}
\end{enum}

正例:
\begin{leftbar}
\begin{c++} 
bool needPassword = true;
bool userIsAuthenticated = true; 
\end{c++}
\end{leftbar}

\begin{regulation}
丰富你的单词库,在面对具体问题时你具有更多的\ascii{Colorfull Words}
\end{regulation}

如\reftbl{colorful-words}所示,同一概念是有很多种表达方式的。

\begin{table}[!htb]
\resizebox{0.95\textwidth}{!} {
\begin{tabular*}{1.2\textwidth}{@{}ll@{}}
\toprule
\ascii{Word} & \ascii{Alternatives} \\
\midrule
\ascii{send}  & \ascii{deliver, dispatch, announce, distribute, route} \\
\ascii{find} & \ascii{dsearch, extract, locate, recover} \\ 
\ascii{start} & \ascii{launch, create, begin, open} \\
\ascii{make} & \ascii{create, set up, build, generate, compose, add, new} \\
\bottomrule
\end{tabular*}
}
\caption{丰富的词汇}
\label{tbl:colorful-words}
\end{table}

\begin{regulation}
名字在明确意图的前提下越短越好
\end{regulation}

反例:
\begin{leftbar}
\begin{c++}
ControllerForEfficientHandlingOfStrings
ControllerForEfficientStorageOfStrings
\end{c++}
\end{leftbar}

因为这两个类名实在是太相似了,一时间很难区分。当别人打算复用你的代码时,必然承受着巨大的记忆包袱。

另外,实体名称的长度应该于作用域的大小成正比。如果是一个实体暴露在全局命名空间,则需要适当增加名字长度,防止名字冲突\footnote{应该使用命名空间,避免增加模块前缀信息。}。相反地,在一个很小的函数之内可见的局部变量,尤其在一个短小的\ascii{for}循环作用域内,完全没有必要取一个很长的名字。

反例:
\begin{leftbar}
\begin{c++}[caption={\ttfamily{cub/trans-dsl/listener/MutilEventListener.cpp}}]
void MutilEventListener::onEventDone()
{
    for(int index=0; index!=listeners.size(); index++)
    {
        listeners[index].onEventDone();
    }
}
\end{c++}
\end{leftbar}

优秀的程序员在命名的长短的取舍总是游刃有余,但从来没有降低过代码的意图。

正例:
\begin{leftbar}
\begin{c++}[caption={\ttfamily{cub/trans-dsl/listener/MutilEventListener.cpp}}]
void MutilEventListener::onEventDone()
{
    for(int i=0; i!=listeners.size(); i++)
    {
        listeners[i].onEventDone();
    }
}
\end{c++}
\end{leftbar}

使用\ascii{C++11}的\ascii{for-each}特性,可进一步改善设计。

正例:
\begin{leftbar}
\begin{c++}[caption={\ttfamily{cub/trans-dsl/listener/MutilEventListener.cpp}}]
void MutilEventListener::onEventDone()
{
    for(auto& listener : listeners)
    {
        listeners.onEventDone();
    }
}
\end{c++}
\end{leftbar}


\begin{regulation}
程序中每个实体都应该有一个意图明确\ascii{(Intention-Revealing)}的名称
\end{regulation}

反例:
\begin{leftbar}
\begin{c++}[caption={\ttfamily{game/chess/GameBoard.h}}]
vector<vector<int> > getThem() 
{
    vector<vector<int> > list1;
    
    for(auto x : theList)
    {
        if(x[0] == 4)
        {   
            list1.add(x);
        }
    }
    
    return list1;
}
\end{c++}
\end{leftbar}

\begin{enum}
  \eitem{\ascii{vector<vector<int> >}的语法令人抓狂}
  \eitem{\ascii{getThem}让人看不清楚它的本意}
  \eitem{\ascii{theList}到底是什么东西?}
  \eitem{\ascii{0}下标意味着什么?\ascii{4}又意味着什么?}
  \eitem{\ascii{list1}就是为了编译通过吗?}
\end{enum}

换一个好名字之后,并对数据结构的实现细节进行了简单的封装,便可以得到一个更合理的设计。

正例:
\begin{leftbar}
\begin{c++}[caption={\ttfamily{game/chess/Cell.h}}]
#ifndef UTJKLFJS_467867_NBHD83562_HETRIO_75649
#define UTJKLFJS_467867_NBHD83562_HETRIO_75649

#include <vector>

struct Cell
{
    bool isFlagged() const;

private:
    std::vector<int> states;
};

using Cells = std::vector<Cell>;

#endif
\end{c++}
\end{leftbar}

\begin{leftbar}
\begin{c++}[caption={\ttfamily{game/chess/GameBoard.h}}]
#ifndef JAAA_1295_BBVBAA_GFYQQPPAAMZZ0092444_N
#define JAAA_1295_BBVBAA_GFYQQPPAAMZZ0092444_N

#include <game/chess/Cell.h>

struct GameBoard
{
    void collectFlaggedCells(Cells&) const;

private:
    Cells board;
};

#endif
\end{c++}
\end{leftbar}

\begin{leftbar}
\begin{c++}[caption={\ttfamily{game/chess/GameBoard.cpp}}]
void GameBoard::collectFlaggedCells(Cells& cells) const
{
    for(auto &cell : board)
    {
        if(cell->isFlagged())
        {
            cells.push_back(cell);
        }
    }
}  
\end{c++}
\end{leftbar}
\end{c++}
\end{leftbar}

\begin{regulation}
避免在名称中携带数据结构的信息
\end{regulation}

\begin{content}

别用\ascii{accountList,accountArray}指定一组账号,当包含\ascii{Account}的容器不在是一个\ascii{List}或\ascii{Array}的时候,就会引发错误的判断。所以,用\ascii{accountGroup},\ascii{bunchOfAccounts},甚至直接使用\ascii{accounts},情况都会更好一些。

\end{content}

\begin{regulation}
\ascii{Noise Words are Redundant},消除噪声后将得到一个更加精准的名字
\end{regulation}

\begin{table}[!htb]
\resizebox{0.95\textwidth}{!} {
\begin{tabular*}{1.2\textwidth}{@{}ll@{}}
\toprule
\ascii{Short Name} & \ascii{Redundant Names} \\
\midrule
\ascii{Name}  & \ascii{StrName, NameString} \\
\ascii{Customer} & \ascii{CustmerObject, CustmerInfo} \\ 
\ascii{accouts} & \ascii{accountList, accountArray} \\
\ascii{accout} & \ascii{accountData, accountInfo} \\  
\ascii{money} & \ascii{moneyAmount} \\
\ascii{message} & \ascii{theMessage} \\
\bottomrule
\end{tabular*}
}
\caption{消除冗余的噪声}
\label{tbl:redundant-words}
\end{table}

\begin{regulation}
使用\ascii{Domain}领域内的名称,将更直白地表明你的设计,增强领域内成员的沟通
\end{regulation}

\begin{enum}
  \eitem{\ascii{Factory, Visitor, Repository}}
  \eitem{\ascii{valueOf, of, getInstance, newInstance, newType}}
  \eitem{\ascii{AppendAble, Closeable, Runnable, Readable, Invokable}}
\end{enum}

\begin{content}

当使用\ascii{Visitor},你在使用访问者模式;当使用\ascii{valueOf,
of},你在使用静态工厂方法。领域内的命名风格,让领域内的成员更加快捷地理解你的设计意图。

\end{content}

\end{content}

\section{匈牙利命名}
\begin{content}

\begin{advise}
摒弃匈牙利命名
\end{advise}

匈牙利命名曾风靡一时。但是,现代编程语言具有更丰富的类型系统;人们更趋于使用更小的类,更短的函数,让每一个变量定义都在可控的视野范围之内;此外,\ascii{IDE}变得越来智能和强大,匈牙利命名反而变成了一种噪声和肉刺。

\begin{advise}
摒弃给成员变量加前缀,或后缀。
\end{advise}

为类的成员变量增加\ascii{m\_}前缀同样没有必要。与其纠结加前缀还是不加前缀,不如花费更多的时间将类分解;当类足够小,职责足够单一,加之现代\ascii{IDE}的着色功能,成员变量和普通变量一眼便能识别开来\footnote{\ascii{Eclipse}默认使用蓝色与普通变量区别开来,其他\ascii{IDE}也提供类似的功能。}。

\begin{advise}
摒弃常量的前缀。
\end{advise}

常量前增加\ascii{k\_}前缀同样没有必要,如果熟悉了大写、下划线隔开的命名风格,\ascii{k\_}前缀完全属于多余。

\begin{advise}
摒弃接口和类的前缀
\end{advise}

在接口前增加\ascii{I\_},在类前增加\ascii{C\_},同样完全没有必要,它只能对重构带来阻力,百害而无一利。

反例:
\begin{leftbar}
\begin{c++}[caption={\ttfamily{trans-dsl/sched/IAction.h}}]
#ifndef NVCHKJSD903085_JGOIUUE906894_VBKSJKJA075774
#define NVCHKJSD903085_JGOIUUE906894_VBKSJKJA075774    

#include <cub/base/Keywords.h>
#include <cub/base/Status.h>

struct TransactionInfo;

DEFINE_ROLE(IAction)
{
    ABSTRACT(Status exec(const TransactionInfo&));
};

#endif
\end{c++}
\end{leftbar}

正例:
\begin{leftbar}
\begin{c++}[caption={\ttfamily{trans-dsl/sched/IAction.h}}]
#ifndef NVCHKJSD903085_JGOIUUE906894_VBKSJKJA075774
#define NVCHKJSD903085_JGOIUUE906894_VBKSJKJA075774    

#include <cub/base/Keywords.h>
#include <cub/base/Status.h>

struct TransactionInfo;

DEFINE_ROLE(Action)
{
    ABSTRACT(Status exec(const TransactionInfo&));
};

#endif
\end{c++}
\end{leftbar}

如果非得在接口与实现中选择,宁愿选择实现中增加\ascii{Impl}后缀。

\end{content}
