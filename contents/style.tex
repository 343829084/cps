\begin{savequote}[45mm]
\ascii{Do the simplest thing that could possibly work.}
\qauthor{\ascii{- Kent Beck}}
\end{savequote}

\chapter{代码风格} 
\label{ch:physical-design}

\section{开发环境}

\begin{content}

\begin{regulation}
为了防止代码出现可移植性问题,团队使用的操作系统、编译器类型、版本保持一致性
\end{regulation}

\begin{regulation}
团队统一使用相同的\ascii{IDE},并使用统一的代码模板,保持代码风格的一致性
\end{regulation}

\begin{regulation}
团队统一配置\ascii{IDE}使用等宽字体,抵制使用宋体编程
\end{regulation}

文档中使用宋体展示示例代码,尤其是使用\ascii{Microsoft Word}时,因为存在各种繁琐的格式化,这无可厚非,除非你是一个追求完美的程序员。

但是,我们处在\ascii{IDE}高度发达的时代,如果团队中还有人使用宋体编码,给团队带来各种代码对不齐、缩进混乱的情况,他的品味及其修养,所提的代码只会给团队的其他人挖坑,再挖坑,不要指望他能给团队带来生产力的提升。

优秀的程序员在写第一行代码前,都会将自己的编译环境、颜色、字体等调整到最佳状态,以便在写代码的时候更友好地、更快捷地反馈所存在的问题,提高工作效率。

\begin{regulation}
团队统一配置\ascii{IDE}的文件编码格式
\end{regulation}

例如统一为\ascii{UTF-8},或\ascii{GBK, GB2312},抵制使用不同的编码,否则中文会出现乱码的现象。

\begin{regulation}
团队统一配置\ascii{IDE}的\ascii{TAB}为相同数目的空格,坚决抵制使用\ascii{TAB}对齐代码\footnote{\ascii{TAB}一般配置为2或4个空格,所有的编辑器都提供了类似的配置接口}
\end{regulation}

因各种编辑器解释\ascii{TAB}的长度存在不一致,如果使用\ascii{TAB}对齐代码,可能会使代码缩进混乱不堪。

\end{content}

\section{代码风格}

\begin{content}

\begin{regulation}
团队应该保持一致的代码风格,例如一致的命名、缩进、空格、空行风格
\end{regulation}

团队代码拥有一种最权威的风格,整个项目犹如一个人写的一样。业界存在多种经典的代码风格,各自都拥有独特的优势,团队应该统一地选择其中一种风格即可。

\begin{enum}
  \eitem{\ascii{K\&R}}
  \eitem{\ascii{BSD/Allman}}
  \eitem{\ascii{GNU}}
  \eitem{\ascii{Whitesmiths}}
\end{enum}

统一代码风格,并非难事,团队发布统一的\ascii{IDE}代码模板,及其定制一个方便的格式化快捷键即可解决所有问题。

\begin{regulation}
程序实体之间有且仅有一行空行区分
\end{regulation}

函数之间的空行,能够帮组我们快速定位函数的始末的准确位置;甚至在函数内部,将逻辑相关的代码放在一起,也同样具有意义,能够帮组我们更好地理解代码的语义。

多余的空行完全没有必要,业余程序员总在这些细节处理中存在着或多或少的问题,团队应该杜绝这样的情况发生。

\begin{regulation}
每个文件末尾都应该有且仅有一行空行,避免编译器警告
\end{regulation}

如果使用\ascii{Eclipse},在创建新文件时,\ascii{Eclipse}会自动在文件末添加一行空行;但诸如\ascii{Visual C++}等情况下,则需要程序员在文件末自行添加一行空行。

\begin{regulation}
在简单逻辑的情况下,鼓励使用\ascii{?:}三元表达式;但是,当表达式变得非常复杂时,\ascii{if-else}可能是更好的选择
\end{regulation}

反例:
\begin{leftbar}
\begin{c++}
if (hour >= 12) 
{
    time += "pm";
} 
else 
{
    time += "am";
}
\end{c++}
\end{leftbar}

正例:
\begin{leftbar}
\begin{c++}
time_str += (hour >= 12) ? "pm" : "am";
\end{c++}
\end{leftbar}

但很多程序员往往因为习惯了\texttt{if-else}而忽视了这种简洁的表达式。但是,当表达式变得非常复杂时,\texttt{if-else}可能是更好的选择。

反例:
\begin{leftbar}
\begin{c++}
return exponent >= 0 ? mantissa * (1 << exponent) : mantissa / (1 << -exponent);
\end{c++}
\end{leftbar}

正例:
\begin{leftbar}
\begin{c++}
if (exponent >= 0)
{
    return mantissa * (1 << exponent);
}
else
{
    return mantissa / (1 << -exponent);
}
\end{c++}
\end{leftbar}

\begin{regulation}
当逻辑表达式已经具有\ascii{true或false}语义时,则无需显示地加上\ascii{true或false}
\end{regulation}

反例:
\begin{leftbar}
\begin{c++}
bool exist = erabs.contains(ErabId(2)) ? true : false;
\end{c++}
\end{leftbar}

正例:
\begin{leftbar}
\begin{c++}
bool exist = erabs.contains(ErabId(2));
\end{c++}
\end{leftbar}

\begin{regulation}
擅用卫语句从函数中提前返回,或从嵌套的语句中提前返回;但绝对不滥用\ascii{break, continue}
\end{regulation}

反例:
\begin{leftbar}
\begin{c++}
bool contains(const std::string* str, const string* substr) 
{
    if (str != NULL && substr != NULL)
    {
        if (!substr->empty())
        {
            ...
        } 
    }
}
\end{c++}
\end{leftbar}

正例:
\begin{leftbar}
\begin{c++}
bool contains(const std::string* str, const string* substr) 
{
    if (str == NULL || substr == NULL)
        return false;
    
    if (substr->empty()) 
        return true;

    ...
}
\end{c++}
\end{leftbar}

虽然在代码阅读过程中\texttt{break, continue, return}常常打断了人的思维。但在函数短小,意图明确的情况下,\texttt{break, continue, return}相反能给人一种“跳过此项”的特殊意图,代码嵌套逻辑得到了改善。

我们赞同\ascii{Dijkstra}提出了结构化编程规则和规范,但对于小函数,这条规则指导意义不是很大,擅用卫语句在此情况下可以改善了代码的清晰度。

\begin{regulation}
擅用解释型变量,或提供意图明确的、内联的提取函数。
\end{regulation}

解释型变量,或提取函数的重构手法有利于改善代码的可读性,往往后者更具有表达力。

反例:
\begin{leftbar}
\begin{c++}
if (line.split(":")[0].trim() == "root")
\end{c++}
\end{leftbar}

正例:
\begin{leftbar}
\begin{c++}
String userName = line.split(":")[0].trim();
if (userName == "root")
{
    ...
}
\end{c++}
\end{leftbar}

\ascii{split}函数将字符串按照指定的正则表达式进行拆分,并提供一个良好的意图变量,将大大改善逻辑的清晰度。

\end{content}
